\documentclass[12pt]{article}

% Packages
\usepackage{amsmath, amssymb}
\usepackage{geometry}
\usepackage{enumerate}

% Page layout settings
\geometry{a4paper, margin=1in}

\begin{document}

% Title section
\title{CS 2340 Assignment 1}
\author{Diego Rodrigues Rodriguez}
\date{\today}
\maketitle

% Questions and Answers
\section*{Question 1}
\textbf{Aside from the smart cell phones used by a billion people, list and describe four other types of computers.}
\begin{enumerate}[a)]
    \item Personal computers (PCs): these are meant to be used by a single person at a time. Designed to be easy to use and affordable.
    \item Supercomputers: these are the most powerful computers in the world. They are used for complex scientific and engineering problems, such as weather forecasting, climate research, and nuclear simulations.
    \item Servers: these are normally stored in data centers for running software applications that serve many users. They are designed to be reliable and scalable.
    \item Embedded computers: these are small, lightweight computers that are built into other products and are dedicated to a specific task. They are used in cars, airplanes, and consumer electronics.
\end{enumerate}

\section*{Question 2}
\textbf{Match the seven great ideas in computer architecture to ideas from other fields.}

\begin{enumerate}[a)]
    \item Assembly lines in automobile manufacturing - \textbf{Performance via pipelining}
    \item Suspension bridge cables - \textbf{Dependability via redundancy}
    \item Aircraft and marine navigation systems that incorporate wind information - \textbf{Performance via prediction}
    \item Express elevators in buildings - \textbf{Common case fast}
    \item Library reserve desk - \textbf{Hierarchy of memories}
    \item A weather simulation model that runs on a supercomputer - \textbf{Performance via parallelism}
    \item Building self-driving cars - \textbf{Use abstraction to simplify design}
\end{enumerate}

\section*{Question 3}
\textbf{Assume a color display using 8 bits for each of the primary colors (red, green, blue) per pixel
and a frame size of 1280 x 1024}
\begin{enumerate}[a)]
    \item \textbf{What is the size of the frame buffer in bytes?} We have $1280 \times 1024 = 1,310,720$ pixels. Each pixel uses 24 bits, so the frame buffer size is $1,310,720 \times 24 / 8 = 3,932,160$ bytes.
    \item \textbf{How long does it take to transmit the frame buffer over a 100 Mbit/s network?} it takes $3,932,160\text{ bytes} \times 8\text{ bits} / 100,000,000 = 315$ seconds.
\end{enumerate}

\section*{Question 4}
\textbf{Assume a 15cm diameter wafer has a cost of 12, contains 84 dies, and has 0.020 defects$/cm^2$. \\ Assume a 20 cm diameter wafer has a cost of 15, contains 100 dies, and has 0.031 defects/$cm^2$.}

First wafer Area: $A = \pi \times (15/2)^2 = 176.71 cm^2$ \\
Second wafer Area $B =\pi \times (20/2)^2 = 314.16 cm^2$

\begin{enumerate}[a)]
    \item \textbf{Find the yield for both wafers} The yield is given by $Y = \frac{1}{(1 + (\text{Defects per area $\times$ Die area/2)}}$. Then:
        \begin{equation*}
            Die\ area\ A = \frac{A}{84} = 2.10 cm^2 \\ 
            Die\ area\ B = \frac{B}{100} = 3.14 cm^2
        \end{equation*}
\end{enumerate}

\end{document}
