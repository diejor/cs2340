\documentclass[12pt]{article}

% Packages
\usepackage{amsmath, amssymb}
\usepackage{geometry}
\usepackage{enumerate}

% Page layout settings
\geometry{a4paper, margin=1in}

\begin{document}

% Title section
\title{CS 2340 Assignment 2}
\author{Diego Rodrigues Rodriguez}
\date{\today}
\maketitle

\section*{Question 1}
\begin{enumerate}
    \item Convert to binary:
        \begin{enumerate}[I.]
            \item $2483_{10} = 100110110011_2$
            \item $3E8A_{16} = 0011\ 1110\ 1000\ 1010_2$
    \end{enumerate}
    \item Convert 8-bit binary to decimal:
        \begin{enumerate}[I.]
            \item $11101011 = 235_{10}$ and $-11101011 = 00010101 \therefore -21_{10}$
            \item $10000000 = 128_{10}$ and $-10000000 = 10000000 \therefore -128_{10}$
            \item $01000101 = 69_{10}$ and $-01000101 = 10111011 \therefore -69_{10}$
        \end{enumerate}
\end{enumerate}

\section*{Question 2}
Do the following addition exercises by translating the numbers into 8-bit 2's complement binary
numbers, performing the arithmetic, and translating the result back into a decimal number.
Indicate where overflow occurs and why, based on the binary arithmetic:
\begin{enumerate}[a.]
    \item $47_{10} + 38_{10} = 0010\ 1111 + 0010\ 0110 = 0101\ 0101_{2} = 85_{10}$
    \item $47_{10} - 38_{10} = 0010\ 1111 - 1101\ 1010 = 0010\ 1111 + 1101\ 1010 = 1000\ 1001_{2} = -9_{10}$
\end{enumerate}

\end{document}

