\documentclass{article}

\title{Homework 3}
\author{Diego Rodrigues}
\date{\today}

\usepackage{amsmath}

\begin{document}
\maketitle

\section{Question 1}

Assume the function declaration for func is “int func (int a, int b);” The code for function f is as
follows:
\begin{verbatim}
    Int f (int a, int b, int c, int d) {
        Return func (func (a, b), c + d)
    }
\end{verbatim}

\begin{enumerate}
    \item Translating function f into MIPS assembly language: if you need to use registers $t0 through $t7, use the lower-number registers first.
    \item Can we use the tail-call optimization in this function? If no, explain why not. If yes, what is the difference in the number of executed instructions in f with and without the optimization?
    \item Right before function f returns, what do we know about contents of registers $t5, $s3, $ra, and $sp? Keep in mind that we know what the entire function f looks like, but for function func we only know its declaration.
\end{enumerate}

\subsection{Answer a}
    
\begin{verbatim}
    f:
        addi    $sp, $sp, -20
        sw      $ra, 0($sp)
        sw      $a0, 4($sp)
        sw      $a1, 8($sp)
        sw      $a2, 12($sp)
        sw      $a3, 16($sp)
        # func is assumed to preserve arguments
        jal func # func (a, b) 
        move    $a0, $v0
        add     $a1, $a2, $a3
        jal func # func (func (a, b), c + d)
        lw      $a3, 16($sp)
        lw      $a2, 12($sp)
        lw      $a1, 8($sp)
        lw      $a0, 4($sp)
        lw      $ra, 0($sp)
        addi    $sp, $sp, 20
        # notice that the return value is in $v0
        jr      $ra
\end{verbatim}

\subsection{Answer b}
    Tail-call optimization cannot be used in this function because f does not call itself.

\subsection{Answer c}
    Right before function f returns, we know the following about the contents of registers $t5, $s3, $ra, and $sp:
    \begin{itemize}
        \item $t5: We do not know what is in $t5, but we know that it is not preserved by f.
        \item $s3: We do not know what is in $s3, is the c argument of f.
        \item $ra: We know that $ra contains the return address of the function f.
        \item $sp: We know that $sp points to the top of the stack frame of the function f.
    \end{itemize}

\section{Question 2}
Write a program in MIPS assembly language to convert an ASCII number string containing integer
decimal strings, to an integer. Your program should expect register \$a0 to hold the address of a
null-terminated string containing some combination of the digits 0 through 9. Your program should
compute the integer value equivalent to this string of digits, then place the number in register \$v0.
If a non-digit character appears anywhere in the string, your program should stop with the value -1
in register \$v0. For example, if register \$t9 points to a sequence of three bytes $50_{10}$ , $52_{10}$ , $0_{10}$ (the
null-terminated string “24”), then when the program stops, register \$v0 should contain the value
$24_{10}$.

\begin{verbatim}
    strtoint:
        addi    $sp, $sp, -8
        sw      $ra, 0($sp)
        sw      $a0, 4($sp)
        li      $v0, 0
        li      $t0, $zero, 0x30 # ASCII '0'
        li      $t1, $zero, 0x39 # ASCII '9'
    lp:
        lb      $t2, 0($a0)
        beq     $t2, $zero, done # check for null terminator

        # check non-digit
        blt     $t2, $t0, error 
        bgt     $t2, $t1, error

        ## convert to integer
        sub     $t2, $t2, $t0
        mul     $v0, $v0, 10
        add     $v0, $v0, $t2
        addi    $a0, $a0, 1 # move to next character
        j       lp

    error:
        li      $v0, -1
        j       done

    done:
        lw      $a0, 4($sp)
        lw      $ra, 0($sp)
        addi    $sp, $sp, 8
        jr      $ra
\end{verbatim}

\section{Question 3}
\begin{enumerate}
    \item Write the MIPS assembly code that creates the 32-bit constant
$0010 0000 0000 0001 0100 1001 0010 0100_2$
and stores that value to register \$t1.
    \item If the current value of the PC is $00000000_{16}$ , can you use a single jump instruction to get
to the PC address as shown by the 32-bit constant in a.? Explain.
    \item If the current value of the PC is $00000600_{16}$ , can you use a single branch instruction to
get to the PC address as shown by the 32-bit constant in a.? Explain.
    \item If the current value of the PC is $1FFFF000_{16}$ , can you use a single branch instruction to
get to the PC address as shown by the 32-bit constant in a.?
\end{enumerate}

\subsection{Answer a}
\begin{verbatim}
    lui     $t1, 0x2001
    ori     $t1, $t1, 0x4924

\end{verbatim}

\end{document}
