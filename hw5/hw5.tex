\documentclass{article}

\title{Homework 5}
\author{Diego Rodrigues}
\date{\today}

\usepackage{amsmath}

\begin{document}
\maketitle

\section{Problem 1}
\begin{enumerate}
    \item[a.] \textbf{How many 32-bit integers can be stored in a 16-byte cache block?}
    Each 32-bit integer occupies 4 bytes:
    \[
    \frac{16 \text{ bytes}}{4 \text{ bytes/integer}} = 4 \text{ integers}.
    \]
    
    \item[b.] \textbf{References to which variables exhibit temporal locality?}
    Temporal locality is exhibited by the loop variables \verb|I|, \verb|J| and addresses of \verb|A| and \verb|B| since they are accessed repeatedly in every iteration of the loop. Also \verb|B(I, 0)| is accessed multiple times in the inner loop and therefore exhibits temporal locality, only in the inner loop as \verb|I| changes in the outer loop and breaks temporal locality.
    
    \item[c.] \textbf{References to which variables exhibit spatial locality?}
    Spatial locality is exhibited by the variables \verb|A(J, I)| it is accessed contiguously in memory because arrays are stored in row-major order. Notice that \verb|A(J, I)| is accessed in column-major which doesnt exhibit spatial locality in C. And \verb|B(I, 0)| is not accessed contiguously as it is jumping row by row in the outer loop.

    \item[d.] \textbf{How many 16-byte cache blocks are needed to store all 32-bit matrix elements being referenced?}
    Given that the matrix has 8 rows and 8000 columns, and each element is a 32-bit integer (4 bytes), the total number of cache blocks required is:
    \[
    \frac{8 \times 8000 \times 4 \text{ bytes}}{16 \text{ bytes/block}} = 16000 \text{ blocks}.
    \]
    
    \item[e.] \textbf{References to which variables exhibit temporal locality?}
    Temporal locality is exhibited by the variables \verb|I|, \verb|J| and addresses of \verb|A| and \verb|B|. Also \verb|B(I, 0)| is accessed multiple times in the inner loop and therefore exhibits temporal locality, only in the inner loop as \verb|I| changes in the outer loop and breaks temporal locality.
    
    \item[f.] \textbf{References to which variables exhibit spatial locality?}
    This time \verb|A(J, I)| and \verb|B(I, 0)| exhibits spatial locality, as it is accessed contiguously through the outer loop. In this case \verb|A(I, J)| is not accessed contiguously as it will jump in memory because of the column-major order.
\end{enumerate}


\section{Problem 2}
\begin{enumerate}
    \item[a.] 
    \begin{table}[h!]
    \centering
    \begin{tabular}{|c|c|c|c|}
    \hline
    Binary Address & Tag & Index & Hit/Miss \\
    \hline
    00000011 & 0000 & 0011 & miss \\
    10110100 & 1011 & 0100 & miss \\
    00101011 & 0010 & 1011 & miss \\
    00000010 & 0000 & 0010 & miss \\
    10111111 & 1011 & 1111 & miss \\
    01011000 & 0101 & 1000 & miss \\
    10111110 & 1011 & 1110 & miss \\
    00001110 & 0000 & 1110 & miss \\
    10110101 & 1011 & 0101 & miss \\
    10110101 & 1011 & 0101 & miss \\
    00101100 & 0010 & 1100 & miss \\
    10111010 & 1011 & 1010 & miss \\
    11111101 & 1111 & 1101 & miss \\
    \hline
    \end{tabular}
    \caption{Cache A: Direct-mapped cache with 16 one-word blocks}
    \end{table}

    \item[b.] 
    \begin{table}[h!]
    \centering
    \begin{tabular}{|c|c|c|c|}
    \hline
    Binary Address & Tag & Index & Hit/Miss \\
    \hline
    00000011 & - & 0001 & miss \\
    10110100 & 101 & 1010 & miss \\
    00101011 & 1 & 0101 & miss \\
    00000010 & - & 0001 & hit \\
    10111111 & 101 & 1111 & miss \\
    01011000 & 10 & 1100 & miss \\
    10001110 & 100 & 0111 & miss \\
    00001110 & - & 0111 & miss \\
    00001110 & - & 0111 & hit \\
    10110101 & 101 & 1010 & hit \\
    00101100 & 1 & 0110 & miss \\
    10111010 & 101 & 1101 & miss \\
    11111101 & 111 & 1110 & miss \\
    \hlin
    \end{tabular}
    \caption{Cache B: Direct-mapped cache with two-word blocks and a total of 8 blocks}
    \end{table}
\end{enumerate}


% Problem 3
\section{Problem 3}
\begin{enumerate}
\item[a.] The cache block size is \textbf{4 words}.
\item[b.] The cache has \textbf{32 entries}.
\item[c.] The ratio between total bits required for such a cache implementation over the data storage bits is approximately \textbf{1.1796875}.

\item{d.}
\begin{table}[h!]
\centering
\begin{tabular}{|c|c|c|c|c|c|}
\hline
Hex Address & Tag & Index & Offset & Hit/Miss & Replaced Tag \\
\hline
00 & 0000000000000000000000 & 00000 & 0000 & miss & - \\
04 & 0000000000000000000000 & 00000 & 0100 & hit & - \\
10 & 0000000000000000000000 & 00000 & 0000 & hit & - \\
84 & 0000000000000000000000 & 00100 & 0100 & miss & - \\
E8 & 0000000000000000000000 & 00111 & 1000 & miss & - \\
A0 & 0000000000000000000000 & 00101 & 0000 & miss & - \\
400 & 0000000000000000000001 & 00000 & 0000 & miss & 0000000000000000000000 \\
1E & 0000000000000000000000 & 00000 & 1110 & miss & 0000000000000000000001 \\
8C & 0000000000000000000000 & 00100 & 1100 & hit & - \\
C1C & 0000000000000000000011 & 00000 & 1100 & miss & 0000000000000000000000 \\
B4 & 0000000000000000000000 & 00101 & 0100 & hit & - \\
884 & 0000000000000000000010 & 00100 & 0100 & miss & 0000000000000000000000 \\
\hline
\end{tabular}
\caption{Cache actions for each address reference.}
\end{table}

\item[e.] The hit ratio for the given cache references is \textbf{0.3333} (or \textbf{33.33\%}).

\item[f.]
\begin{itemize}
    \item $<$00000, 0000000000000000000011$>$
    \item $<$00100, 0000000000000000000010$>$
    \item $<$00101, 0000000000000000000000$>$
    \item $<$00111, 0000000000000000000000$>$
\end{itemize}
\end{enumerate}


\section*{Question 4 Answers}

\begin{enumerate}

\item[a.]
\begin{tabular}{|c|c|}
\hline
Word Address & Hit/Miss \\
\hline
2 & miss \\
3 & miss \\
11 & miss \\
16 & miss \\
21 & miss \\
13 & miss \\
64 & miss \\
48 & miss \\
19 & miss \\
11 & hit \\
3 & miss \\
22 & miss \\
4 & miss \\
27 & miss \\
6 & miss \\
11 & miss \\
\hline
\end{tabular}

Final cache contents:
\begin{tabular}{|c|c|}
\hline
Set Index & Words \\
\hline
0 & 64, 48 \\
1 & - \\
2 & 2 \\
3 & 27, 11 \\
4 & 4 \\
5 & 21, 13 \\
6 & 22, 6 \\
7 & - \\
\hline
\end{tabular}

\item[b.]
\begin{tabular}{|c|c|}
\hline
Word Address & Hit/Miss \\
\hline
2 & miss \\
3 & miss \\
11 & miss \\
16 & miss \\
21 & miss \\
13 & miss \\
64 & miss \\
48 & miss \\
19 & miss \\
11 & hit \\
3 & hit \\
22 & miss \\
4 & miss \\
27 & miss \\
6 & miss \\
11 & hit \\
\hline
\end{tabular}

Final cache contents:
\begin{tabular}{|c|}
\hline
Words \\
\hline
2, 3, 11, 16, 21, 13, 64, 48, 19, 22, 4, 27, 6 \\
\hline
\end{tabular}

\end{enumerate}

\end{document}
